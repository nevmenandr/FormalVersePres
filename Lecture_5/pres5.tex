%----------------------------------------------------------------------------------------
%	PACKAGES AND THEMES
%----------------------------------------------------------------------------------------

\documentclass{beamer}

\mode<presentation> {

\usetheme{Madrid}

}


\usepackage{graphicx} % Allows including images
\usepackage{booktabs} % Allows the use of \toprule, \midrule and \bottomrule in tables

\usepackage{verbatim}

%%% Работа с русским языком
\usepackage[T2A]{fontenc}			% кодировка
\usepackage[LGR,T1]{fontenc}
\usepackage[utf8]{inputenc}			% кодировка исходного текста
\usepackage[english, russian]{babel}	% локализация и переносы

%%% Работа с картинками
\setlength\fboxsep{3pt} % Отступ рамки \fbox{} от рисунка
\setlength\fboxrule{1pt} % Толщина линий рамки \fbox{}
\usepackage{wrapfig} % Обтекание рисунков текстом

%%% Оформление стихов
\usepackage{verse}

\usepackage{philex}

%%% Зачёркивания
\usepackage{ulem}

%%% Параллельные тексты
\usepackage{parallel}


\AtBeginSection[]
{
  \begin{frame}
    \frametitle{Содержание}
    \tableofcontents[currentsection]
  \end{frame}
}




%----------------------------------------------------------------------------------------
%	TITLE PAGE
%----------------------------------------------------------------------------------------

\title[Занятие 5]{Формальный анализ стиха. Занятие 5} % The short title appears at the bottom of every slide, the full title is only on the title page

\author{Борис Орехов} % Your name
\institute[НИУ ВШЭ] % Your institution as it will appear on the bottom of every slide, may be shorthand to save space
{
НИУ Высшая школа экономики \\ % Your institution for the title page
\medskip
\textit{nevmenandr@gmail.com} % Your email address
}
\date{6 октября 2015} % Date, can be changed to a custom date

\begin{document}

\begin{frame}
\titlepage % Print the title page as the first slide
\end{frame}



\begin{frame}
\frametitle{Содержание}  % Table of contents slide, comment this block out to remove it
\tableofcontents % Throughout your presentation, if you choose to use \section{} and \subsection{} commands, these will automatically be printed on this slide as an overview of your presentation
\end{frame}

%----------------------------------------------------------------------------------------
%	PRESENTATION SLIDES
%----------------------------------------------------------------------------------------

\section{Русская силлабо-тоника}\label{sec:rusyl}

%------------------------------------------------
\begin{frame}
\frametitle{Русские размеры}
\begin{center}
\textbf{Двусложные размеры}
\end{center}

\begin{itemize}
\item  $\smile$ --- ямб
\item  --- $\smile$ хорей
\end{itemize}

\begin{center}
\textbf{Трехсложные размеры}
\end{center}

\begin{itemize}
\item  --- $\smile$ $\smile$ дактиль
\item  $\smile$ --- $\smile$ амфибрахий
\item  $\smile$ $\smile$ --- анапест
\end{itemize}

\end{frame}



%------------------------------------------------



\begin{frame}
\frametitle{Метры и ритмы}

\begin{itemize}
\item Метр: идеальная схема: \\
\begin{verse}
$\smile$ --- $\mid$ $\smile$ --- $\mid$ $\smile$ --- $\mid$ $\smile$ ---
\end{verse}
\item Ритм: реализация схемы:\\
\begin{verse}
 Когд\'{а} не в ш\'{у}тку занем\'{о}г\\
 $\smile$ --- $\mid$ $\smile$ --- $\mid$ $\smile$ $\smile$ $\mid$ $\smile$ ---
\end{verse}
\item Форма: разновидность реализации:
\begin{itemize}
\item I форма $\smile$ --- $\mid$ $\smile$ --- $\mid$ $\smile$ --- $\mid$ $\smile$ ---
\item II форма $\smile$ $\smile$ $\mid$ $\smile$ --- $\mid$ $\smile$ --- $\mid$ $\smile$ ---
\item III форма $\smile$ --- $\mid$ $\smile$ $\smile$ $\mid$ $\smile$ --- $\mid$ $\smile$ ---
\end{itemize}
\end{itemize}


\end{frame}


\begin{frame}
\frametitle{Метры и ритмы}

\textbf{Я4жм}
\begin{verse}
Филолог некий, муж науки,\\
Боготворя свою жену,\\
Готов был на любые муки,\\
Её чтоб радовать одну.\\
Но не переставая злиться\\
На благоверного, она\\
Изволила отворотиться —\\
И неудовлетворена.
\end{verse}

А. А. Илюшин, 1996

\end{frame}


\begin{frame}
\frametitle{Пропуск ударений}

\begin{verse}
Весна, я с улицы, где тополь удивлен, \\
Где даль пугается, где дом упасть боится,\\ 
Где воздух синь, как узелок с бельем \\
\alert{У выписавшегося из больницы}. 

Где вечер пуст, как прерванный рассказ,\\ 
Оставленный звездой без продолженья \\
К недоуменью тысяч шумных глаз, \\
Бездонных и лишенных выраженья. 
\end{verse}

Б. Пастернак, 1918

\end{frame}

%------------------------------------------------


%------------------------------------------------
\section{Русская поэзия XVIII века}\label{sec:sys} % Sections can be created in order to organize your presentation into discrete blocks, all sections and subsections are automatically printed in the table of contents as an overview of the talk
%------------------------------------------------

\begin{frame}
\frametitle{Поэты XVIII века в~Русской виртуальной библиотеке}

\begin{center}
{\LARGE http://rvb.ru/18vek/}
\end{center}

\end{frame}

%

\begin{frame}
\frametitle{Крупнейшие поэты XVIII века}

\begin{itemize}
\item Василий Иванович Майков (1728\,--\,1778), автор пародийной поэмы «Елисей, или Раздраженный Вакх», варьирующей сюжет «Энеиды».
\item Михаил Матвеевич Херасков (1733\,--\,1807), популярный среди современников автор эпической поэмы «Россиада».
\item Яков Борисович Княжнин (1740\,--\,1791), издание его~трагедии «Вадим Новгородский» было~сожжено по~приказу Екатерины~II.
\item Ипполит Федорович Богданович (1743\,--\,1803), автор иронической поэмы «Душенька», стихотворного переложения романа Ж.",Лафонтена «Любовь Психеи и~Купидона».
\item Александр Николаевич Радищев (1749\,--\,1802), самая знаменитая его~книга "--- «Путешествие из~Петребурга в Москву», но~оды «Вольность» и~«Осьмнадцатое столетие» были ценимы и~современниками, и~Пушкиным.
\end{itemize}

\end{frame}

\begin{frame}
\frametitle{Крупнейшие поэты XVIII века}

\begin{itemize}
\item Иван Иванович Хемницер (1745\,--\,1784), крупнейший русский поэт"=баснописец второй половины XVIII века (а~вовсе не~Крылов).
\item Михаил Никитич Муравьев (1757\,--\,1807), один из~первых сентиментальных поэтов в~России.
\item Алексей Фёдорович Мерзляков (1778\,--\,1830), поэт и~профессор Московского университета, переводчик.
\item Иван Иванович Дмитриев (1760\,--\,1837), наряду с~Н.",М.",Карамзиным Дмитриев "--- один из~крупнейших представителей русского сентиментализма.
\item Николай Михайлович Карамзин (1766\,--\,1826), в~XVIII веке автор популярных сентиментальных повестей и~поэт, позднее известен как~официальный придворный историк.
\item Иван Андреевич Крылов (1769\,--\,1844), до~басен, благодаря которым стал популярен в~XIX веке, писал сатирическую прозу и~комедии.
\end{itemize}

\end{frame}


\section{Сумароков}\label{sec:sum}

%------------------------------------------------

\begin{frame}
\frametitle{Александр Петрович Сумароков (1717\,--\,1777)}

Один из~трёх, вместе с~Ломоносовым и~Тредиаковским, основателей русской поэтической традиции в~XVIII веке. Втроём они сначала дружили, а~потом враждовали. Автор басен и~пьес для~театра, основателем которого в~России он~может считаться. Метрически значительно разнообразнее Ломоносова.

\end{frame}

%------------------------------------------------

\begin{frame}
\frametitle{А. П. Сумароков, всегда актуальные стихи}
\begin{center}
\textbf{Эпиграмма}
\end{center}

\textbf{Я6м}
\begin{verse}
Танцовщик! Ты богат. Профессор! Ты убог.\\
Конечно, голова в почтеньи меньше ног.
\end{verse}
<1759>
\end{frame}

%

\begin{frame}
\frametitle{Логаэд}

\textbf{Л4м}, 0*2*1*1*0
\begin{verse}
Где ни гуляю, ни хожу,	\\
Грусть превеликую терплю;\\
Скучно мне, где я ни сижу,\\
Лягу, спокойно я не сплю;\\
Нет мне веселья никогда.\\
Горько мне, горько завсегда,\\
Сердце мое тоска щемит,\\
С грусти без памяти бегу;\\
Грудь по тебе моя болит,\\
Вся по тебе я немогу;\\
Ты завсегда в моих глазах,\\
Я по тебе всегда в слезах, "---\\
То ли не лютая беда! <\dots>
\end{verse}
<1765>
\end{frame}

%

\begin{frame}
\frametitle{}

\begin{center}
\textbf{Другой хор ко превратному свету}
\end{center}

\begin{verse}
Приплыла к нам на берег собака\\
Прилетела на берег синица\\
Из"=за полночного моря,\\
Из"=за холодна океяна.\\
Спрашивали гостейку приезжу,\\
За морем какие обряды.\\
Гостья приезжа отвечала:\\
«Всё там превратно на свете.\\
За морем Сократы добронравны,\\
Каковых мы здесь <не> видаем,\\
Никогда не суеверят,\\
Не ханжат, не лицемерят,\\
Воеводы за морем правдивы;\\
Дьяк там цуками не ездит,\\
Дьячихи алмазов не носят,\\
Дьячата гостинцев не просят,\\
За нос там судей писцы не водят.
\end{verse}
\end{frame}


\begin{frame}
\frametitle{}

\begin{verse}
Сахар подьячий покупает.\\
За морем подьячие честны,\\
За морем писать они умеют.\\
За морем в подрядах не крадут;\\
Откупы за морем не в моде,\\
Чтобы не стонало государство.\\
«Завтрем» там истца не питают.\\
За морем почетные люди\\
Шеи назад не загибают,\\
Люди от них не погибают.\\
В землю денег за морем не прячут,\\
Со крестьян там кожи не сдирают,\\
Деревень на карты там не ставят,\\
За морем людьми не торгуют.\\
За морем старухи не брюзгливы,\\
Четок они хотя не носят,\\
Добрых людей не злословят.\\
За морем противно указу\\
Росту заказного не емлют.
\end{verse}
\end{frame}


\begin{frame}
\frametitle{}

\begin{verse}

За морем пошлины не крадут.\\
В церкви за морем кокетки\\
Бредить, колобродить не ездят.\\
За морем бездельник не входит,\\
В домы, где добрые люди.\\
За морем людей не смучают,\\
Сору из избы не выносят.\\
За морем ума не пропивают;\\
Сильные бессильных там не давят;\\
Пред больших бояр лампад не ставят,\\
Все дворянски дети там во школах,\\
Их отцы и сами учились;\\
Учатся за морем и девки;\\
За морем того не болтают:\\
Девушке"=де разума не надо,\\
Надобно ей личико да юбка,\\
Надобны румяна да белилы.
\end{verse}
\end{frame}


\begin{frame}
\frametitle{}

\begin{verse}
Там язык отцовский не в презреньи;\\
Только в презреньи те невежи,\\
Кои свой язык уничтожают,\\
Кои, долго странствуя по свету,\\
Чужестранным воздухом некстати\\
Головы пустые набивая,\\
Пузыри надутые вывозят.\\
Вздору там ораторы не мелют;\\
Стихотворцы вирши не кропают;\\
Мысли у писателей там ясны,\\
Речи у слагателей согласны:\\
За морем невежа не пишет,\\
Критика злобой не дышит;\\
Ябеды за морем не знают,\\
Лучше там достоинство "--- наука,\\
Лучше приказного крюка.
\end{verse}
\end{frame}


\begin{frame}
\frametitle{}

\begin{verse}
Хитрости свободны там почтенней,\\
Нежели дьячьи закрепы,\\
Нежели выписки и справки,\\
Нежели невнятные экстракты.\\
Там купец "--- купец, а не обманщик.\\
Гордости за морем не терпят,\\
Лести за морем не слышно,\\
Подлости за морем не видно.\\
Ложь там! "--- велико беззаконье.\\
За морем нет тунеядцев.\\
Все люди за морем трудятся,\\
Все там отечеству служат;\\
Лучше работящий там крестьянин,\\
Нежель господин тунеядец;\\
Лучше нерасчесаны кудри,\\
Нежели парик на болване.
\end{verse}
\end{frame}


\begin{frame}
\frametitle{}

\begin{verse}
За морем почтеннее свиньи,\\
Нежели бесстыдны сребролюбцы,\\
За морем не любятся за деньги:\\
Там воеводская метресса\\
Равна своею степенью\\
С жирною гадкою крысой.\\
Пьяные по улицам не ходят,\\
И людей на улицах не режут>>.\\

\end{verse}
Конец 1762\,--\,январь 1763
\end{frame}

\begin{frame}
\frametitle{}

\textbf{Л3мж}
\begin{verse}
Благополучны дни\\
Нашими временами;\\
Веселы мы одни,\\
Хоть нет и женщин с нами:\\
Честности здесь уставы,\\
Злобе, вражде конец,\\
Ищем единой славы\\
От чистоты сердец.

Гордость, источник бед,\\
Распрей к нам не приводит,\\
Споров меж нами нет,\\
Брань нам и в ум не входит;\\
Дружба, твои успехи\\
Увеселяют нас;\\
Вот наши все утехи,\\
Благословен сей час.

\end{verse}
<1730"=е годы>
\end{frame}

%------------------------------------------------


\begin{frame}
\frametitle{}

О.~Мандельштам

\begin{verse}
Есть ценностей незыблемая скала\\
Над скучными ошибками веков.\\
Неправильно наложена опала\\
На автора возвышенных стихов.

И вслед за тем, как жалкий Сумароков\\
Пролепетал заученную роль,\\
Как царский посох в скинии пророков,\\
У нас цвела торжественная боль.

Что делать вам в театре полуслова\\
И полумаск, герои и цари?\\
И для меня явленье Озерова "---\\
Последний луч трагической зари. 

\end{verse}
1914

\end{frame}

\section{Державин}\label{sec:d}

%------------------------------------------------

\begin{frame}
\frametitle{Гавриил (Гаврила) Романович Державин}

\begin{flushleft}
3 (14) июля 1743, село Сокуры, Лаишевского уезда Казанской губернии (ныне село Державино Лаишевского района Татарстана) — 8~(20) июля~1816, имение Званка, Новгородская губерния
\end{flushleft}

\begin{flushleft}
Русский поэт и~государственный деятель, воевавший под~началом А.",В.",Суворова. Приобрёл известность и~благосклонность императрицы благодаря оде «Фелица», рисующей Екатерину~II человечным правителем. Открыл русским поэтам возможность совмещения противоположностей в~стилях и~жанрах. Уволен Александром~I с~поста министра юстиции с~формулировкой «Ты~очень ревностно служишь».
\end{flushleft}

\end{frame}

%------------------------------------------------


\begin{frame}
\frametitle{}
\begin{center}
\textbf{Снигирь}
\end{center}

\textbf{Д2ж$ \backsim $м|Д$ \backsim $Аф2жм; Д4м}

\begin{verse}
\hspace{2em}Что ты заводишь песню военну\\
Флейте подобно, милый снигирь?\\
С кем мы пойдем войной на Гиену?\\			
Кто теперь вождь наш? Кто богатырь?\\
Сильный где, храбрый, быстрый Суворов?\\
Северны громы в гробе лежат.\\
		\hspace{2em}Кто перед ратью будет, пылая,\\
Ездить на кляче, есть сухари;\\
В стуже и в зное меч закаляя,\\
Спать на соломе, бдеть до зари;\\
Тысячи воинств, стен и затворов\\
С горстью россиян все побеждать?


\end{verse}


\end{frame}

%------------------------------------------------

\begin{frame}
\frametitle{}

\begin{verse}
		\hspace{2em}Быть везде первым в мужестве строгом;\\
Шутками зависть, злобу штыком,\\
Рок низлагать молитвой и Богом,	\\		
Скиптры давая, зваться рабом;\\
Доблестей быв страдалец единых,\\			
Жить для царей, себя изнурять?	\\		
		\hspace{2em}Нет теперь мужа в свете столь славна:\\
Полно петь песню военну, снигирь!	\\
Бранна музыка днесь не забавна,\\
Слышен отвсюду томный вой лир;\\
Львиного сердца, крыльев орлиных\\
Нет уже с нами! "--- что воевать?
\end{verse}

 Май 1800
\end{frame}

%------------------------------------------------

\begin{frame}
\frametitle{}
Иосиф Бродский
\begin{center}
\textbf{На смерть Жукова}
\end{center}

\begin{verse}
Вижу колонны замерших звуков,\\
гроб на лафете, лошади круп.\\
Ветер сюда не доносит мне звуков\\
русских военных плачущих труб.\\
Вижу в регалиях убранный труп:\\
в смерть уезжает пламенный Жуков.

Воин, пред коим многие пали\\
стены, хоть меч был вражьих тупей,\\
блеском маневра о Ганнибале\\
напоминавший средь волжских степей.\\
Кончивший дни свои глухо в опале,\\
как Велизарий или Помпей.
\end{verse}
\end{frame}


\begin{frame}
\frametitle{}

\begin{verse}

Сколько он пролил крови солдатской\\
в землю чужую! Что ж, горевал?\\
Вспомнил ли их, умирающий в штатской\\
белой кровати? Полный провал.\\
Что он ответит, встретившись в адской\\
области с ними? «Я воевал».

К правому делу Жуков десницы\\
больше уже не приложит в бою.\\
Спи! У истории русской страницы\\
хватит для тех, кто в пехотном строю\\
смело входили в чужие столицы,\\
но возвращались в страхе в свою.
\end{verse}
\end{frame}


\begin{frame}
\frametitle{}

\begin{verse}
Маршал! поглотит алчная Лета\\
эти слова и твои прахоря.\\
Все же, прими их "--- жалкая лепта\\
родину спасшему, вслух говоря.\\
Бей, барабан, и военная флейта,\\
громко свисти на манер снегиря.


\end{verse}
1974
\end{frame}

%------------------------------------------------


\begin{frame}
\frametitle{}

\begin{Parallel}{157pt}{}
\ParallelLText{
\textbf{Р}ека времен в своем стремленье

\textbf{У}носит все дела людей

\textbf{И} топит в пропасти забвенья

\textbf{Н}ароды, царства и царей.

\textbf{А} если что и остается

\textbf{Ч}рез звуки лиры и трубы, "--- 

\textbf{Т}о вечности жерлом пожрется

\textbf{И} общей не уйдет судьбы!}
\ParallelRText{\textbf{A}ufert fugaci temporis impetu

{\small \textbf{M}ortalia amnis gesta hominum omnia~et}

\textbf{O}blivionis mergit alto

\textbf{R}egna duces populos hiatu.

\textbf{S}i quae supersunt quomodo gratia

\textbf{T}ubaeve vocis gratave per lyram, "---

\textbf{A}eternitatis devorantur

\textbf{T}urbine; fata eadem cuivis.
}
\end{Parallel}
\begin{flushleft}
6 июля 1816
\end{flushleft}

\begin{flushleft}
Перевод Т.",В.",Васильевой~// Традиция в~истории культуры. М.: Наука, 1978. С.~178.
\end{flushleft}

\end{frame}

\begin{frame}
\frametitle{}
\begin{center}
\textbf{Ласточка}
\end{center}

\begin{verse}
О домовитая Ласточка!\\	
О милосизая птичка!	\\			
Грудь красно-бела, касаточка,\\			
Летняя гостья, певичка!			\\	
Ты часто по кровлям щебечешь,\\
Над гнездышком сидя, поешь,\\
Крылышками движешь, трепещешь,\\		
Колокольчиком в горлышке бьешь.	\\	
Ты часто по воздуху вьешься,\\
В нем смелые круги даешь;\\
Иль стелешься долу, несешься,\\
Иль в небе простряся плывешь.\\
Ты часто во зеркале водном\\
Под рдяной играешь зарей,\\
На зыбком лазуре бездонном\\
Тенью мелькаешь твоей.
\end{verse}
\end{frame}


\begin{frame}
\frametitle{}

\begin{verse}
Ты часто, как молния, реешь\\
Мгновенно туды и сюды;\\
Сама за собой не успеешь\\
Невидимы видеть следы, "---\\
Но видишь там всю ты вселенну,\\
Как будто с высот на ковре:\\
Там башню, как жар позлащенну,\\
В чешуйчатом флот там сребре;\\
Там рощи в одежде зеленой,\\
Там нивы в венце золотом,\\
Там холм, синий лес отдаленный,\\
Там мошки толкутся столпом;\\
Там гнутся с утеса в понт воды,\\
Там ластятся струи к брегам.\\
Всю прелесть ты видишь природы,\\
Зришь лета роскошного храм;\\
Но видишь и бури ты черны,\\
И осени скучной приход;
\end{verse}
\end{frame}


\begin{frame}
\frametitle{}

\begin{verse}
И прячешься в бездны подземны,\\
Хладея зимою, как лед.\\
Во мраке лежишь бездыханна, "---\\
Но только лишь придет весна\\
И роза вздохнет лишь румяна,\\
Встаешь ты от смертного сна;\\
Встанешь, откроешь зеницы	\\	
И новый луч жизни ты пьешь;\\
Сизы оправя косицы,			\\	
Ты новое солнце поешь\dots
\end{verse}
1792

\end{frame}



%------------------------------------------------

\begin{frame}
\Huge{\centerline{продолжение следует}}
\end{frame}

\end{document}