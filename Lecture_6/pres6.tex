%----------------------------------------------------------------------------------------
%	PACKAGES AND THEMES
%----------------------------------------------------------------------------------------

\documentclass{beamer}

\mode<presentation> {

\usetheme{Madrid}

}


\usepackage{graphicx} % Allows including images
\usepackage{booktabs} % Allows the use of \toprule, \midrule and \bottomrule in tables

\usepackage{verbatim}

%%% Работа с русским языком
\usepackage[T2A]{fontenc}			% кодировка
\usepackage[LGR,T1]{fontenc}
\usepackage[utf8]{inputenc}			% кодировка исходного текста
\usepackage[english, russian]{babel}	% локализация и переносы

%%% Работа с картинками
\setlength\fboxsep{3pt} % Отступ рамки \fbox{} от рисунка
\setlength\fboxrule{1pt} % Толщина линий рамки \fbox{}
\usepackage{wrapfig} % Обтекание рисунков текстом

%%% Оформление стихов
\usepackage{verse}

\usepackage{philex}

%%% Зачёркивания
\usepackage{ulem}

%%% Параллельные тексты
\usepackage{parallel}


\AtBeginSection[]
{
  \begin{frame}
    \frametitle{Содержание}
    \tableofcontents[currentsection]
    %\tableofcontents[currentsubsection]
  \end{frame}
}

\AtBeginSubsection[]
{
   \begin{frame}
        \frametitle{Содержание}
        \tableofcontents[currentsection,currentsubsection]
   \end{frame}
}




%----------------------------------------------------------------------------------------
%	TITLE PAGE
%----------------------------------------------------------------------------------------

\title[Занятие 6]{Формальный анализ стиха. Занятие 6} % The short title appears at the bottom of every slide, the full title is only on the title page

\author{Борис Орехов} % Your name
\institute[НИУ ВШЭ] % Your institution as it will appear on the bottom of every slide, may be shorthand to save space
{
НИУ Высшая школа экономики \\ % Your institution for the title page
\medskip
\textit{nevmenandr@gmail.com} % Your email address
}
\date{13 октября 2015} % Date, can be changed to a custom date

\begin{document}

\begin{frame}
\titlepage % Print the title page as the first slide
\end{frame}



\begin{frame}
\frametitle{Содержание}  % Table of contents slide, comment this block out to remove it
\tableofcontents % Throughout your presentation, if you choose to use \section{} and \subsection{} commands, these will automatically be printed on this slide as an overview of your presentation
\end{frame}

%----------------------------------------------------------------------------------------
%	PRESENTATION SLIDES
%----------------------------------------------------------------------------------------

\section{Изученные понятия силлабо-тоники}\label{sec:rusyl}

%------------------------------------------------
\begin{frame}
\frametitle{Русские размеры}
\begin{center}
\textbf{Двусложные размеры}
\end{center}

\begin{itemize}
\item  $\smile$ --- ямб
\item  --- $\smile$ хорей
\end{itemize}

\begin{center}
\textbf{Трехсложные размеры}
\end{center}

\begin{itemize}
\item  --- $\smile$ $\smile$ дактиль
\item  $\smile$ --- $\smile$ амфибрахий
\item  $\smile$ $\smile$ --- анапест
\end{itemize}

\end{frame}



%------------------------------------------------



\begin{frame}
\frametitle{Метры и ритмы}

\begin{itemize}
\item Метр: идеальная схема: \\
\begin{verse}
$\smile$ --- $\mid$ $\smile$ --- $\mid$ $\smile$ --- $\mid$ $\smile$ ---
\end{verse}
\item Ритм: реализация схемы:\\
\begin{verse}
 Когд\'{а} не в ш\'{у}тку занем\'{о}г\\
 $\smile$ --- $\mid$ $\smile$ --- $\mid$ $\smile$ $\smile$ $\mid$ $\smile$ ---
\end{verse}
\item Форма: разновидность реализации
\item Метр подсказывается метрической инерцией
\item Логаэды
\end{itemize}


\end{frame}


\begin{frame}
\frametitle{Стопность, метр и клаузула}

Полное описание строки состоит из трёх параметров:

\begin{itemize}
\item метр
\item стопность
\item клаузула
\end{itemize}

Клаузула "--- число безударных слогов после последнего ударного.

\begin{itemize}
\item м, мужская
\item ж, женская
\item д, дактилическая
\item г, гипердактилическая
\end{itemize}

\end{frame}


\begin{frame}
\frametitle{Попробуем}

\begin{center}
\textbf{Сельское кладбище}
\end{center}

\begin{verse}
Уже бледнеет день, скрываясь за горою;\\
Шумящие стада толпятся над рекой;\\
Усталый селянин медлительной стопою\\
Идет, задумавшись, в шалаш спокойный свой.

В туманном сумраке окрестность исчезает...\\
Повсюду тишина; повсюду мертвый сон;\\
Лишь изредка, жужжа, вечерний жук мелькает,\\
Лишь слышится вдали рогов унылый звон.

Лишь дикая сова, таясь под древним сводом\\
Той башни, сетует, внимаема луной,\\
На возмутившего полуночным приходом\\
Ее безмолвного владычества покой.
\end{verse}

В. А. Жуковский, 1802

\end{frame}


\begin{frame}

\begin{center}
\textbf{Суд Божий над епископом}
\end{center}

\begin{verse}
Были и лето и осень дождливы;\\
Были потоплены пажити, нивы;\\
Хлеб на полях не созрел и пропал;\\
Сделался голод; народ умирал.

Вот уж столпились под кровлей сарая\\
Все пришлецы из окружного края...\\
Как же их принял епископ Гаттон?\\
Был им сарай и с гостями сожжен.

Зубы об камни они навострили,\\
Грешнику в кости их жадно впустили,\\
Весь по суставам раздернут был он...\\
Так был наказан епископ Гаттон.
\end{verse}

В. А. Жуковский, 1831

\end{frame}


\begin{frame}

\begin{center}
\textbf{смерть философа}
\end{center}

\begin{verse}
вянут подснежники сохнет береза\\
грустно на свете без жиля делеза\\
то есть без жака нам жить деррида\\
был бодрийяр но и этот туда

меркнут созвездия гаснут светила\\
снова костлявая жертву схватила\\
всех в поминальную вносим тетрадь\\
жижека тоже пора потерять

даже мудрейший живет не по триста\\
матрица в трауре смерть сценариста\\
слезы сторицей и долг платежом\\
так мы философов не бережем
\end{verse}

\end{frame}

\begin{frame}
\begin{verse}
отбыл лакан где фуко ему пара\\
как же нам жить теперь без лиотара\\
только остался один хабермас\\
грустную песню заводит для нас

разум в гробу на погосте харизма\\
нет больше классиков постмодернизма\\
франция плачь марианна умри\\
их не заменят бордо или бри

\end{verse}
Алексей Цветков, 2007
\end{frame}

\begin{frame}

\begin{center}
\textbf{Светлана}
\end{center}

\begin{verse}
Раз в крещенский вечерок\\
Девушки гадали:\\
За ворота башмачок,\\
Сняв с ноги, бросали;\\
Снег пололи; под окном\\
Слушали; кормили\\
Счётным курицу зерном;\\
Ярый воск топили;\\
В чашу с чистою водой\\
Клали перстень золотой,\\
Серьги изумрудны;\\
Расстилали белый плат\\
И над чашей пели в лад\\
Песенки подблюдны.

\end{verse}

В. А. Жуковский, 1808\,--\,1812

\end{frame}
%------------------------------------------------

\begin{frame}

\begin{center}
\textbf{Цветок}
\end{center}

\begin{verse}

Минутная краса полей,\\
Цветок увядший, одинокой,\\
Лишён ты прелести своей\\
Рукою осени жестокой.

	Увы! нам тот же дан удел,\\
И тот же рок нас угнетает:\\
С тебя листочек облетел –\\
От нас веселье отлетает.

	Отъемлет каждый день у нас\\
Или мечту, иль наслажденье.\\
И каждый разрушает час\\
Драгое сердцу заблужденье.


\end{verse}

В. А. Жуковский, 1811

\end{frame}

\begin{frame}

\begin{center}
\textbf{Голос с того света}
\end{center}

\begin{verse}
Не узнавай, куда я путь склонила,\\
В какой предел из мира перешла...\\
О друг, я всё земное совершила;\\
Я на земле любила и жила.

	Нашла ли их? Сбылись ли ожиданья?\\
Без страха верь; обмана сердцу нет;\\
Сбылося всё; я в стороне свиданья;\\
И знаю здесь, сколь ваш прекрасен свет.

	Друг, на земле великое не тщетно;\\
Будь тверд, а здесь тебе не изменят;\\
О милый, здесь не будет безответно\\
Ничто, ничто: ни мысль, ни вздох, ни взгляд.
\end{verse}

В. А. Жуковский, 1815

\end{frame}


\begin{frame}

\begin{center}
\textbf{Море}
\end{center}

\begin{verse}

Безмолвное море, лазурное море,\\
Стою очарован над бездной твоей.\\
Ты живо; ты дышишь; смятенной любовью,\\
Тревожною думой наполнено ты.\\
Безмолвное море, лазурное море,\\
Открой мне глубокую тайну твою.\\
Что движет твоё необъятное лоно?\\
Чем дышит твоя напряженная грудь?\\
Иль тянет тебя из земныя неволи\\
Далёкое, светлое небо к себе?..\\
Таинственной, сладостной полное жизни,\\
Ты чисто в присутствии чистом его\dots


\end{verse}

В. А. Жуковский, 1822

\end{frame}


\begin{frame}

\begin{center}
\textbf{Жалоба пастуха}
\end{center}

\begin{verse}
На ту знакомую гору\\
Сто раз я в день прихожу;\\
Стою, склоняся на посох,\\
И в дол с вершины гляжу.\\
Вздохнув, медлительным шагом\\
Иду вослед я овцам\\
И часто, часто в долину\\
Схожу, не чувствуя сам.\\
Весь луг по-прежнему полон\\
Младой цветов красоты;\\
Я рву их  – сам же не знаю,\\
Кому отдать мне цветы.

\end{verse}

В. А. Жуковский, 1817

\end{frame}


%------------------------------------------------
\section{Пеоны}\label{sec:sys} % Sections can be created in order to organize your presentation into discrete blocks, all sections and subsections are automatically printed in the table of contents as an overview of the talk
%------------------------------------------------

\begin{frame}
\frametitle{Стопы другой длины?}

\begin{verse}
Как я стал знать взор твой,\\
С тех пор мой дух рвет страсть;\\
С тех пор весь сгиб сон мой;\\
Стал знать с тех пор я власть.

\end{verse}

А. Ржевский <<Ода, собранная из односложных слов>>, 1761

\begin{verse}
Били копыта.\\
Пели будто:\\
— Гриб.\\
Грабь.\\
Гроб.\\
Груб. —

\end{verse}
В. В. Маяковский «Хорошее отношение к лошадям», 1918

\end{frame}

%
\subsection{Пеон I}

\begin{frame}
\frametitle{Пеон I}
\begin{verse}
Ирисы печальные, задумчивые, бледные,\\
Сказки полусонные неведомой страны!\\
Слышите ль дыхание ликующе-победное\\
Снова возвратившейся, неснившейся весны?

Слышите ль рыдания снежинок, голубеющих\\
Под лучами знойными в бездонной высоте?\\
Видите ль сверкание небес, мечту лелеющих\\
Вечною мелодией о вечной красоте?

Нет! Вы, утомленные, поникли — и не знаете,\\
Как звенит — алмазами пронизанная даль…\\
Только скорбь неясную вы тихо вызываете,\\
Только непонятную, стыдливую печаль.
\end{verse}
Н.Г. Львова, 1912
\end{frame}

\begin{frame}

\begin{center}
\textbf{Придорожные травы }
\end{center}

\begin{verse}
Спите, полумертвые увядшие цветы, \\
Так и не узнавшие расцвета красоты, \\
Близ путей заезженных взращенные творцом, \\
Смятые невидевшим тяжелым колесом. 

В час, когда все празднуют рождение весны, \\
В час, когда сбываются несбыточные сны, \\
Всем дано безумствовать, лишь вам одним нельзя, \\
Возле вас раскинулась заклятая стезя. 

Вот, полуизломаны, лежите вы в пыли, \\
Вы, что в небо дальнее светло глядеть могли, \\
Вы, что встретить счастие могли бы, как и все, \\
В женственной, в нетронутой, в девической красе. 

\end{verse}
К. Бальмонт, 1900

\end{frame}


\subsection{Пеон II}\label{sec:sum}

%------------------------------------------------

\begin{frame}
\frametitle{Пеон II}

\begin{center}
\textbf{Фонарики}
\end{center}

\begin{verse}
Фонарики-сударики, \\
Скажите-ка вы мне, \\
Что видели, что слышали \\
В ночной вы тишине? \\
Так чинно вы расставлены \\
По улицам у нас. \\
Ночные караульщики,\\ 
Ваш верен зоркий глаз! 
\end{verse}

И. П. Мятлев, 8 ноября 1841

\end{frame}

%------------------------------------------------

\begin{frame}

\begin{verse}
От полюса до полюса я землю обошёл,\\
Я плыл путями водными, и счастья не нашёл.

Я шёл один пустынями, я шёл во тьме лесов,\\
И всюду слышал возгласы мятежных голосов.

И думал я, и проклял я бездушие морей,\\
И к людям шёл, и прочь от них в простор бежал скорей.

Где люди, там поруганы виденья высших грёз,\\
Там тление, скрипение назойливых колёс.

О, где ж они, далёкие невинности года,\\
Когда светила сказочно вечерняя звезда?

\end{verse}

К. Д. Бальмонт, 1901/1902

\end{frame}

%
\subsection{Пеон III}\label{sec:sum3}

\begin{frame}
\frametitle{Пеон III}

\begin{verse}
Как за р\alert{е}ченькой слоб\alert{о}душка сто\alert{и}т,\\
По слоб\alert{о}дке той дор\alert{о}женька беж\alert{и}т,\\
Путь-дор\alert{о}жка широк\alert{а}, да не длинн\alert{а},\\
Разбег\alert{а}ется в две ст\alert{о}роны он\alert{а}:\\
Как налево — на кладбище к мертвецам,\\
А направо — к закавказским молодцам,\\
Грустно было провожать мне, молодой,\\
Двух родимых и по той, и по другой:\\
Обручальника по левой проводя,\\
С плачем матерью-землей покрыла я;\\
А налетный друг уехал по другой,\\
На прощанье мне кивнувши головой.
\end{verse}
А.А. Дельвиг, 1828
\end{frame}

%

\begin{frame}


\begin{verse}
Хорош\alert{а} была Тан\alert{ю}ша, краше н\alert{е} было в сел\alert{е},\\
Красной р\alert{ю}шкою по б\alert{е}лу сараф\alert{а}н на подол\alert{е}.\\
У овр\alert{а}га за плетн\alert{я}ми ходит Т\alert{а}ня ввечер\alert{у}.\\
Месяц в \alert{о}блачном тум\alert{а}не водит с т\alert{у}чами игр\alert{у}.\\
Вышел парень, поклонился кучерявой головой:\\
«Ты прощай ли, моя радость, я женюся на другой».\\
Побледнела, словно саван, схолодела, как роса.\\
Душегубкою-змеею развилась ее коса.\\
«Ой ты, парень синеглазый, не в обиду я скажу,\\
Я пришла тебе сказаться: за другого выхожу».\\
Не заутренние звоны, а венчальный переклик,\\
Скачет свадьба на телегах, верховые прячут лик.
\end{verse}
С. А. Есенин, 1911
\end{frame}

\subsection{Пеон IV}\label{sec:sum4}


\begin{frame}
\begin{center}
\textbf{Сборный пункт}
\end{center}

\begin{verse}
На Петербургской стороне в стенах военного училища\\
Столичный люд притих и ждет, как души бледные чистилища. \\
Сгрудясь пугливо на снопах, младенцев кормят грудью женщины, –\\
Что горе их покорных глаз пред тёмным грохотом военщины?.. \\
Ковчег-манеж кишит толпой. Ботфорты чавкают и хлюпают. \\
У грязных столиков врачи нагое мясо вяло щупают. \\
Над головами в полумгле проносят баки с дымной кашею. \\
Оторопелый пиджачок, крестясь, прощается с папашею… \\
Скользят галантно писаря, – бумажки треплются под мышками, \\
В углу – невинный василёк – хохочет девочка с мальчишками. \\

\end{verse}
Саша Чёрный, 1914
\end{frame}


\begin{frame}
\frametitle{}

\begin{verse}

Из-за чего, из-за кого в солдаты взяты\\
Все милоюноши в расцвете вешних лет?\\
Из-за чего, из-за кого все нивы смяты\\
И поразвеян нежных яблонь белоцвет?\\
Из-за чего, из-за кого взят я в солдаты,\\
Я, ваш изысканный, изнеженный поэт?

И люди ль – люди? Ах, не люди – кровопийцы.\\
Все человечество потоками льет кровь.\\
Все англичане, все французы, все бельгийцы\\
Все в исступлении восстали за любовь!\\
И только гению в солдаты не годится,\\
Уже отдавшему всю жизнь свою за новь!..
\end{verse}

И. Северянин, 1915
\end{frame}



\section{Гиперпеоны}\label{sec:d}

%------------------------------------------------

\begin{frame}
\frametitle{Даниил Леонидович Андреев}

\begin{flushleft}
2 ноября 1906, Берлин—30 марта 1959, Москва
\end{flushleft}

\begin{flushleft}
Русский поэт и писатель, автор мистического сочинения «Роза Мира». Второй сын известного русского писателя Леонида Николаевича Андреева (1871—1919) и внучатой племянницы Тараса Шевченко Александры Михайловны Андреевой (урожд. Велигорской; 1881—1906). Работал художником-оформителем. Арестован в 1947 г., освобожден в 1957 г., большую часть заключения провел во Владимирской тюрьме.
\end{flushleft}

\end{frame}

%------------------------------------------------


\begin{frame}
\begin{center}
\textbf{Гипер-пэон}
\end{center}

\begin{verse}
О три\alert{у}мфах, иллюмин\alert{а}циях, гекат\alert{о}мбах,		2*4*4*1\\
Об ов\alert{а}циях всенар\alert{о}дному пал\alert{а}чу,					2*4*4*0\\
О пог\alert{и}бших и погиб\alert{а}ющих в катак\alert{о}мбах \\
Нержав\alert{е}ющий и нез\alert{ы}блемый стих ищ\alert{у}.

Не подск\alert{а}жут мне закат\alert{и}вшиеся эп\alert{о}хи\\
Злу всем\alert{и}рному соотв\alert{е}тствующий разм\alert{е}р,\\
Не пом\alert{о}гут – во всеохв\alert{а}тывающем взд\alert{о}хе –\\
Ритмом в\alert{ы}разить велич\alert{а}йшую из хим\alert{е}р.\\

Её поступью оглушённому, что мне томный\\
Тенор ямба с его усадебною тоской?\\
Я работаю, чтоб улавливали потомки\\
Шаг огромнее и могущественнее, чем людской.

\end{verse}


\end{frame}

%------------------------------------------------

\begin{frame}

\begin{verse}
Её поступью оглушённому, что мне томный\\
Тенор ямба с его усадебною тоской?\\
Я работаю, чтоб улавливали потомки\\
Шаг огромнее и могущественнее, чем людской.

Чтобы в грузных, нечеловеческих интервалах\\
Была тяжесть, как во внутренностях Земли,\\
Ход чудовищ, необъяснимых и небывалых,\\
Из-под магмы приподнимающихся вдали

За расчерченною, исследованною сферой,\\
За последнею спондеической крутизной,\\
Сверхтяжёлые, трансурановые размеры\\
В мраке медленно поднимаются предо мной.

\end{verse}


\end{frame}

%------------------------------------------------

\begin{frame}

\begin{verse}
Опрокидывающий правила, как плутоний,\\
Зримый будущим поколеньям, как пантеон,\\
Встань же, грубый, неотшлифованный, многотонный,\\
Ступенями нагромождаемый сверх-пэон!

Не расплавятся твои сумрачные устои,\\
Не прольются перед кумирами, как елей!\\
Наши судороги под расплющивающей пятою,\\
Наши пытки и наши казни запечатлей!

И свидетельство о склонившемся к нашим мукам\\
Уицраоре, угашающем все огни, \\
Ты преемникам — нашим детям — и нашим внукам —\\
Как чугунная усыпальница, сохрани. 
\end{verse}
1951
\end{frame}


\begin{frame}

\begin{center}
\textbf{Железная мистерия}
\end{center}

\begin{verse}

Я не знаю, какой воскуривать Тебе ладан				2*4*4*1\\
И какие Тебе присваивать имена.						2*4*4*0\\
Только сердцем благоговеющим Ты угадан,			2*4*4*1\\
Только встреча с Твоим сиянием предрешена.		2*4*5*0

Твои тихие, расколдовывающие силы					2*4*5*1\\
Отмыкают с неукоснительностью часов				2*4*4*0\\
Слух мой, замкнутый от колыбели и до могилы,	2*5*4*1\\
Зренье, запертое от рождения на засов.				2*5*4*0

Совлекаемая невидимыми перстами,\\
Всё прозрачнее истончающаяся  ткань,\\
И мерцает за ней – не солнце ещё, не пламя,\\
Но восходу его предшествующая рань.
\end{verse}
\end{frame}


\begin{frame}
\frametitle{}

\begin{verse}
Поднимаешься предварениями до храмов\\
На вершинах многонародных метакультур,\\
Ловишь эхо перводыхания Парабрамы\\
В столкновении  мирозданий и брамфатур.

И я чувствую в потрясающие мгновенья,\\
Что за гранью и галактической, и земной,\\
Ты нас примешь, как сопричастников вдохновенья\\
Для сотворчества и сорадования  с Тобой.

И, разгадывая  вечнодвижущиеся знаки\\
На скрижалях метаистории и судьбы,\\
Различаю и в мимолетном, как в Зодиаке,\\
Те же ходы миропронизывающей борьбы.


\end{verse}
\end{frame}

%------------------------------------------------


\begin{frame}
\begin{verse}
Дух замедливает у пламенного порога:\\
Он прислушивается, он вглядывается в грозу,\\
В обнаруживаемый замысл  Противобога,\\
В цитадели его владычества – там, внизу;

Он возносит свою надежду и упованья\\
К ослепительнейшим соборам Святой Руси,\\
Что в годину непредставимого ликованья\\
Отразятся на земле, как на небеси.

Катастрофам и планетарным преображеньям "---\\
Первообразам, приоткрывшимся вдалеке "---\\
Я зеркальности обрету ли без искаженья\\
В этих строфах на человеческом языке?
\end{verse}
\end{frame}

\begin{frame}

\begin{verse}
Опрокинутся общепризнанные каноны,\\
Громоздившиеся веками, как пантеон;\\
В стих низринутся "--- полнозвучны и многозвонны "---\\
Первенствующие спондей и гиперпеон.

И, не зная ни успокоенья, ни постоянства,\\
Странной лексики обращающаяся праща\\
Разбросает добросозвучья и диссонансы,\\
Непреклонною диалектикой скрежеща.

Не отринь же меня за бред и косноязычье,\\
Небывалое это Действо благослови,\\
Ты, Чьему  благосозиданию и величью\\
Мы сыновствуем во творчестве и любви.

\end{verse}
1956
\end{frame}


%------------------------------------------------

\begin{frame}
\Huge{\centerline{продолжение следует}}
\end{frame}

\end{document}